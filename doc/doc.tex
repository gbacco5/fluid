\documentclass[b5paper,11pt,oneside,fleqn]{article}
\usepackage{amsmath,amsthm,amssymb}

\usepackage{geometry}

\title{\raggedleft\sffamily\Huge fluid}
\author{\sffamily\itshape Bacco, Giacomo}
\date{}

\newcommand{\eu}{\mathrm{e}}
\newcommand{\je}{\mathrm{j}}

\newcommand{\ped}[1]{_\textup{#1}}
\newcommand{\ap}[1]{^\textup{#1}}
\newcommand{\trib}[1][]{t\ped{rib#1}}

\begin{document}
\maketitle

\section{Flow past a cylinder}

Let $ \rho_0 $ be the radius of the cylinder, $ \rho,\xi $ the polar coordinate 
system in use.
One possible solution of this problem have these potential and streamline 
functions:
\begin{align}
\phi(\rho,\xi) &= \left( \rho + \frac{\rho_0^2}{\rho} \right) \cos\xi \\[1ex]
\psi(\rho,\xi) &= \left( \rho - \frac{\rho_0^2}{\rho} \right) \sin\xi 
\end{align}
Although these equations are deeply coupled, the radius $ \rho $ and the phase 
$ \xi $ can be obtained as a function of the other quantities.
For our purposes, we use $ \psi $.
\begin{align}
\rho(\psi,\xi) &= \frac{\psi + \sqrt{\psi^2 + 4\rho_0^2 \sin^2\xi}}{2\sin\xi} 
\\[1ex]
\xi(\psi,\rho) &= \arcsin \left( \frac{\rho\, \psi}{\rho^2 - \rho_0^2} \right)  
\end{align}



\section{Conformal mapping}
From the reference plane, which is equivalent to a two-pole machine,
we use a complex map to obtain the quantities in the actual plane.
Let $ p $ be the number of pole pairs. Then:
\begin{equation}
\begin{aligned}
\zeta              &\xrightarrow{\;\mathcal{M}\;} z = \sqrt[p]{\zeta} \\
\rho\,\eu^{\je\xi} &\xrightarrow{\;\mathcal{M}\;}
                       r \, \eu^{\je\vartheta} = \sqrt[p]{\rho}\,\eu^{\je 
                       \xi/p} \\
\chi + \je\eta     &\xrightarrow{\;\mathcal{M}\;} x + \je y 
\end{aligned}
\end{equation}
%
It is easy to find the inverse map:
\begin{equation}
\mathcal{M}\colon \sqrt[p]{\cdot} \qquad 
\mathcal{M}^{-1}\colon (.)^p
\end{equation}



\section{Computation of flux-barrier base points}
Let the flux-barrier and flux-carrier thicknesses be given.
Then the base points for the flux-barriers can be computed easily.





\subsection{Magnet insertion}
\[
\trib = \trib + w\ped{m}
\]
where $ w\ped{m} $ is the magnet width.


\subsection{Central base points}
The line describing the $ q $-axis is
\begin{equation}
\begin{aligned}
y &= m x + q \\
m &= \tan\frac{\pi}{2p} \\
q &= \frac{\trib}{\cos\frac{\pi}{2p}}
\end{aligned}
\end{equation}

\begin{equation}
\begin{cases}
y_c - m x_c - q = 0 \\
x_c - r_c(\psi_c,\theta_c) \cos\theta_c = 0 \\
y_c - r_c(\psi_c,\theta_c) \sin\theta_c = 0 \\
\end{cases}
\end{equation}







\end{document}